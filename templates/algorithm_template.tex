%%% Local Variables:
%%% mode: latex
%%% TeX-master: t
%%% End:

\documentclass{article}

\usepackage[ruled]{algorithm2e}
\usepackage{hyperref}
\usepackage{amsmath}
\usepackage{zed-csp}
\usepackage{breqn}
\usepackage{listings}
\lstset{literate = {-}{-}1} % get dashs to show up
%%% Start of text
\begin{document}
\section*{Introduction}
Intro text for the entire document, outline structure and define the purpose
\section{xAPI Data Retrival}
The following section describes how to Query an LRS to retrieve a statement or a set of statements
  \footnote{\label{moreLink} S is the set of all statements parsed from the statements array within the HTTP response to the Curl request. It may be possible that multiple Curl requests are needed to retrieve all query results. If multiple requests are necessary, S is the result of concatenating the result of each request into a single set}
  \begin{lstlisting}[frame=single]
Agent = "agent={"account":
                {"homePage": "https://example.homepage",
                 "name": 123456}}"

Since = "since=2018-07-20T12:08:47Z"

Until = "until=2018-07-21T12:08:47Z"

Base = "https://example.endpoint/statements?"

endpoint = Base + Agent + "&" + Since + "&" + Until

Auth = Hash generated from basic auth

S = curl -X GET -H "Authorization: Auth"
         -H "Content-Type: application/json"
         -H "X-Experience-API-Version: 1.0.3"
         Endpoint
       \end{lstlisting}
\begin{itemize}
  \item Update as needed to reflect the needs of an LRS query
  \end{itemize}


\section{xAPI Z Specifications}
An xAPI statement(s) is only defined abstractly within the context
  of Z. A concrete definition for an xAPI statement(s) it outside the
  scope of this specification.

\subsection{Basic Types}
  $IFI$ ::=$ mbox \,|\, mbox\_sha1sum \,|\, openid \,|\, account$
  \begin{itemize}
  \item Type unique to Agents and Groups, The concrete definition of the listed values
    is outside the scope of this specification
  \end{itemize}
  $OBJECTTYPE$ ::=$ Agent \,|\, Group \,|\, SubStatement \,|\,
  StatementRef \,|\, Activity$
  \begin{itemize}
  \item A type which can be present in all activities as defined by
    the xAPI specification
  \end{itemize}
  $INTERACTIONTYPE$ ::= $true-false \,|\, choice \,|\, fill-in \,|\,
  long-fill-in \,|\, matching \,| \\ performance \,|\, sequencing \,|\,
  likert \,|\, numeric \,|\, other$
  \begin{itemize}
  \item A type which represents the possible interactionTypes as
      defined within the xAPI specification
  \end{itemize}
  $INTERACTIONCOMPONENT$ ::= $choices \,|\, scale \,|\, source \,|\,
  target \,|\, steps$
  \begin{itemize}
  \item A type which represents the possible interaction components as
    defined within the xAPI specification
  \item the concrete definition of the listed values is outside the
    scope of this specification
  \end{itemize}
  $CONTEXTTYPES$ ::= $parent \,|\, grouping \,|\, category \,|\, other$
  \begin{itemize}
  \item A type which represents the possible context types as
    defined within the xAPI specification
  \end{itemize}
  $[STATEMENT]$
  \begin{itemize}
  \item Basic types for the results of querying an LRS
  \end{itemize}
  $[AGENT, GROUP]$
  \begin{itemize}
  \item Basic types for Agents and collections of Agents
  \end{itemize}

  \subsection{Schema}
  Z schema for statements and the components of statements

  \subsubsection{Id Schema}
  \begin{schema}{Id}
    id : \finset_1 \#1
  \end{schema}
  \begin{itemize}
  \item the schema $Id$ introduces the component $id$ which is a
      non-empty finite set of 1 value
  \end{itemize}

  \subsubsection{Schemas for Agents and Groups}

  \begin{schema}{Agent}
    agent : AGENT \\
    objectType : OBJECTTYPE \\
    name : \finset_1 \#1 \\
    ifi : IFI
    \where
    objectType = Agent \\
    agent = \{ifi\} \cup \power \{name, objectType\}
  \end{schema}
  \begin{itemize}
  \item The schema $Agent$ introduces the component $agent$ which is a set
    consisting of an $ifi$ and optionally an $objectType$ and/or $name$
  \end{itemize}

  \begin{schema}{Member}
    Agent \\
    member : \finset_1
    \where
    member = \{a : AGENT \,|\, \forall a: a_{0}..a_{n} @ a = agent\}
  \end{schema}
  \begin{itemize}
  \item The schema $Member$ introduces the component $member$ which is a set of
    objects $a$, where for every $a$ within $a_{0}..a_{n}$, $a$ is an $agent$
  \end{itemize}

  \begin{schema}{Group}
    Member \\
    group : GROUP \\
    objectType : OBJECTTYPE \\
    ifi : IFI
    \where
    name : \finset_1 \#1\\
    objectType = Group \\
    group = \{objectType, name, member\} \lor \{objectType, member\}
    \lor \\ \t2 \{objectType, ifi\} \cup \power \{name, member\}
  \end{schema}
  \begin{itemize}
  \item The schema $Group$ introduces the component $group$ which is of
    type $GROUP$ and is a set of either $objectType$ and $member$ with optionaly $name$ or
    $objectType$ and $ifi$ with optionally $name$ and/or $member$
  \end{itemize}

  \begin{schema}{Actor}
    Agent \\
    Group \\
    actor : AGENT \lor GROUP
    \where
    actor = agent \lor group
  \end{schema}
  \begin{itemize}
  \item The schema $Actor$ introduces the component $actor$ which
    is either an $agent$ or $group$
  \end{itemize}

  \subsubsection{Verb Schema}
  \begin{schema}{Verb}
    Id \\
    display, verb : \finset_1
    \where
    verb = \{id, display\} \lor \{id\}
  \end{schema}
  \begin{itemize}
  \item The schema $Verb$ introduces the component $verb$ which
      is a set that consists of either $id$ and the finite set
      $display$ or just $id$
  \end{itemize}

  \subsubsection{Object Schema}

  \begin{schema}{Extensions}
    extensions, extensionVal : \finset_1 \\
    extensionId : \finset_1 \#1 \\
    \where
    extensions = \{e : (extensionId, extensionVal)\ \,|\,
    \forall i,j : e_{i}..e_{j} @ \\
    \t3 \, (extensionId_{i}, extensionVal_{i})
    \lor (extensionId_{i}, extensionVal_{j}) \land \\
    \t3 \, (extensionId_{j}, extensionVal_{i})
    \lor (extensionId_{j}, extensionVal_{j})
    \land \\ \t3 \, extensionId_{i} \not = extensionId_{j}\}
  \end{schema}
  \begin{itemize}
  \item The schema $Extensions$ introduces the component $extensions$ which
      is a non-empty finite set that consists of ordered pairs of
      $extensionId$ and $extensionVal$. Different $extensionId$s can
      have the same $extensionVal$ but there can not be two identical
      $extensionId$ values
  \item $extensionId$ is a non-empty finite set with one value
  \item $extensionVal$ is a non-empty finite set
  \end{itemize}

  \begin{schema}{InteractionActivity}
    interactionType : INTERACTIONTYPE \\
    correctResponsePattern : \seq_1 \\
    interactionComponent: INTERACTIONCOMPONENT \\
    \where
    interactionActivity = \{interactionType, correctReponsePattern,
    interactionComponent\} \lor \\ \t5 \{interactionType, correctResponsePattern\}
  \end{schema}
  \begin{itemize}
  \item The schema $InteractionActivity$ introduces the component
    $interactionActivity$ which is a set of either $interactionType$
    and $correctResponsePattern$ or $interactionType$ and
    $correctResponsePattern$ and $interactionComponent$
  \end{itemize}

  \begin{schema}{Definition}
    InteractionActivity \\
    Extensions \\
    definition, name, description : \finset_1 \\
    type, moreInfo : \finset_1 \#1
    \where
    definition = \power_1 \{name, description, type, moreInfo,
    extensions, interactionActivity\} \\
  \end{schema}
  \begin{itemize}
    \item The schema $Definition$ introduces the component
      $definition$ which is the non-empty, finite power set of $name$, $description$,
      $type$, $moreInfo$ and $extensions$
  \end{itemize}

  \begin{schema}{Object}
    Id \\
    Definition \\
    Agent \\
    Group \\
    Statement \\
    objectTypeA, objectTypeS, objectTypeSub, objectType  : OBJECTTYPE \\
    substatement : STATEMENT \\
    object : \finset_1 \\
    \where
    substatement = statement \\
    objectTypeA = Activity \\
    objectTypeS = StatementRef \\
    objectTypeSub = SubStatement \\
    objectType = objectTypeA \lor objectTypeS \\
    object = \{id\} \lor \{id, objectType\} \lor \{id, objectTypeA,
    definition\} \\ \t2 \lor \{id, definition\} \lor \{agent\} \lor
    \{group\} \lor \{objectTypeSub, substatement\} \\
    \t2 \lor \{id, objectTypeA\}
  \end{schema}
  \begin{itemize}
    \item The schema $Object$ introduces the component $object$ which
      is a non-empty finite set of either $id$, $id$ and $objectType$,
      $id$ and $objectTypeA$ and $definition$, $agent$, $group$, or
      $substatement$
    \item The schema $Statement$ and the corresponding component
      $statement$ will be defined later on in this specification
  \end{itemize}

  \subsubsection{Result Schema}

  \begin{schema}{Score}
    score : \finset_1 \\
    scaled, min, max, raw : \num \\
    \where
    scaled = \{ n : \num \,|\, -1.0 \leq n \leq 1.0 \} \\
    min = n < max \\
    max = n > min \\
    raw = raw = \{ n : \num \,|\, min \leq n \leq max \} \\
    score = \power_1 \{scaled, raw, min, max\}
  \end{schema}
  \begin{itemize}
  \item The schema $Score$ introduces the component $score$ which is
    the non-empty powerset of $min$, $max$, $raw$ and $scaled$
  \end{itemize}

  \begin{schema}{Result}
    Score \\
    Extensions \\
    success, completion, response, duration : \finset_1 \#1 \\
    result : \finset_1
    \where
    success = true \lor false \\
    completion = true \lor false \\
    result = \power_1 \{score, success, completion, response,
    duration, extensions\}
  \end{schema}
  \begin{itemize}
  \item The schema $Result$ introduces the component $result$ which is
    the non-empty power set of $score$, $success$, $completion$,
    $response$, $duration$ and $extensions$
  \end{itemize}

  \subsubsection{Context Schema}

  \begin{schema}{Instructor}
    Agent \\
    Group \\
    instructor : AGENT \lor GROUP
    \where
    instructor = agent \lor group
  \end{schema}
  \begin{itemize}
  \item The schema $Instructor$ introduces the component $instructor$
    which can be ether an $agent$ or a $group$
  \end{itemize}

  \begin{schema}{Team}
    Group \\
    team : GROUP
    \where
    team = group
  \end{schema}
  \begin{itemize}
  \item The schema $Team$ introduces the component $team$ which is a $group$
  \end{itemize}

  \begin{schema}{Context}
    Instructor \\
    Team \\
    Object \\
    Extensions \\
    registration, revision, platform, language : \finset_1 \#1 \\
    parentT, groupingT, categoryT, otherT : CONTEXTTYPES \\
    contextActivities, statement : \finset_1
    \where
    statement = object \hide (id, objectType, agent, group,
    definition) \\
    parentT = parent \\
    groupingT = grouping \\
    categoryT = category \\
    otherT = other \\
    contextActivity = \{ca : object \hide (agent, group, objectType,
    objectTypeSub, substatement)\} \\
    contextActivityParent = (parentT, contextActivity) \\
    contextActivityCategory = (categoryT, contextActivity) \\
    contextActivityGrouping = (groupingT, contextActivity) \\
    contextActivityOther = (otherT, contextActivity) \\
    contextActivities = \power_1 \{contextActivityParent,
    contextActivityCategory, \\ \t5 \:\: contextActivityGrouping,
    contextActivityOther\} \\
    context = \power_1 \{registration, instructor, team,
    contextActivities, revision, \\ \t3 platform, language, statement, extensions\}
  \end{schema}
  \begin{itemize}
  \item The schema $Context$ introduces the component $context$
    which is the non-empty powerset of $registration$, $instructor$,
    $team$, $contextActivities$, $revision$, $platform$, $language$,
    $statement$ and $extensions$
  \end{itemize}

  \subsubsection{Timestamp and Stored Schema}

  \begin{schema}{Timestamp}
    timestamp : \finset_1 \#1
  \end{schema}

  \begin{schema}{Stored}
    stored : \finset_1 \#1
  \end{schema}
  \begin{itemize}
  \item The schema $Timestamp$ and $stored$ introduce the components
    $timestamp$ and $stored$ respectively. Each are non-empty finite
    sets containing one value
  \end{itemize}

  \subsubsection{Attachements Schema}

  \begin{schema}{Attachments}
    display, description, attachment, attachments: \finset_1 \\
    usageType, sha2, fileUrl, contexntType : \finset_1 \#1 \\
    length : \nat
    \where
    attachment = \{usageType, display, contentType, length, sha2 \}
    \cup \power \{description, fileUrl\} \\
    attachments = \{a : attachment\}
  \end{schema}

  \subsubsection{Statement and Statements Schema}

  \begin{schema}{Statement}
    Id \\
    Actor \\
    Verb \\
    Object \\
    Result \\
    Context \\
    Timestamp \\
    Stored \\
    Attachements \\
    statement,\$ : STATEMENT

    \where
    statement = \{actor, verb, object, stored\} \cup \\\t3 \power \{\id,
    result, context, timestamp, attachments \} \\
    \$ \bind statement
  \end{schema}
  \begin{itemize}
  \item The schema $Statement$ introduces the component $statement$
    which consists of the components $actor$, $verb$, $object$ and
    $stored$ and the optional components $id$, $result$, $context$,
    $timestamp$, and/or $attachments$
  \item The schema $Statement$ also binds the component $statement$ to
    the variable $\$$ so that JSONPath can be used within Operation
    schemas which require reaching into a $statement$. This is
    accomplished by using the $.$ (select) notation starting at $\$$
    (root) and navigating into subsequent components of the $statement$
  \end{itemize}

  \begin{schema}{Statements}
    Statement \\
    statements : \finset_1
    \where
    statements = \{s : statement \}
  \end{schema}
  \begin{itemize}
  \item The schema $Statements$ introduces the component $statements$
    which is a non-empty finite set of components $statement$
 \end{itemize}
\section{Question 1  Name}
intro text for the question

\subsection{Statements}

\subsubsection{Ideal Statements}
paragraph or list describing the ideal input statements

\subsubsection{statement parameters to utilize}

\begin{itemize}
  \item first param
  \item second param
  \item third param
  \end{itemize}

\subsubsection{TLA Statement problems}
paragraph talking about known data issues within current TLA
implementation

\subsection{Algorithm}

\subsubsection{Summary}
\begin{enumerate}
  \item step 1
  \item step 2
  \item step 3
  \end{enumerate}

\subsection{Z Specification}

\subsubsection{Introduce Basic Types}
\begin{paragraph}{Template}
  [Name of variable(s) of type set]
\end{paragraph}
\begin{paragraph}{Example}
  [X]
\end{paragraph}
\subsubsection{Example Schema}
Basic unit of specification, defines state variables, system state, operations, etc.
\begin{paragraph}{Template}
\begin{schema}{Schema Name}
  Variable Declarations
  \where
  Predicate/Invariants
\end{schema}
\end{paragraph}
\begin{paragraph}{Example}
  \begin{schema}{Counter}
  ctx : \nat
  \where
  0 \leq ctr \leq max
\end{schema}
\begin{paragraph}{Variables}
  \begin{schema}{Counter}
    ctx : \nat
    \where
\end{schema}
\begin{itemize}
\item the variable ctx is a natural number
\end{itemize}
\end{paragraph}
\begin{paragraph}{Predicates}
  \begin{schema}{Counter}
  \where
  0 \leq ctr \leq max
\end{schema}
  \begin{itemize}
  \item ctr is greater than or equal to 0
  \item ctr is less than or equal to max
  \end{itemize}
\end{paragraph}
\end{paragraph}

\subsubsection{Initialisation}
The starting conditions
\begin{paragraph}{Template}
\begin{schema}{Init[VarName]}
  NameOfExistingSchema
  \where
  InitStateOfVarsWithinRefSchema
\end{schema}
\end{paragraph}
\begin{paragraph}{Example}
  \begin{schema}{InitCounter}
    Counter
    \where
    ctr = 0
  \end{schema}
  \begin{itemize}
    \item the value of the counter starts at 0
  \end{itemize}
\end{paragraph}

\subsubsection{Operations}
an operation is specified in Z with a predicate relating the state before and after the invocation of that operation
\begin{paragraph}{Template}
  \begin{schema}{OperationName}
    \Delta SchemaName \\
    inputParam? : SomeType \\
    outputParam! : SomeType
    \where
    InvariantPredicate \\
    NewValForVar' = OperationOnInput/OutputParams
  \end{schema}
\end{paragraph}
\begin{paragraph}{Example}
  \begin{schema}{Increment}
    \Delta Counter
    \where
    ctr < max \\
    ctr' = crt + 1
  \end{schema}
  \begin{itemize}
    \item There is an implicit conjunction (logical-and) between successive lines of the predicate
    \end{itemize}
    \begin{schema}{Decrement}
      \Delta Counter \\
      d? : \nat
      \where
      ctr \geq d? \\
      ctr' = ctr - d?
    \end{schema}
    \begin{itemize}
    \item input params suffixed with ?
    \end{itemize}
    \begin{schema}{Display}
      \Xi Counter \\
      c! : \nat
      \where
      c! = ctr
    \end{schema}
    \begin{itemize}
    \item output params suffixed with !
    \item the greek symbol means that the operation cannot change the state of Counter
    \end{itemize}
  \end{paragraph}
  \subsection{Pseudocode}
  \begin{algorithm}[H]
    \SetAlgoLined
    \KwIn{this text}
    \KwResult{how to write algorithm with \LaTeX2e }
    initialization\;
    \While{not at end of this document}{
      read current\;
      \eIf{understand}{
        go to next section\;
        current section becomes this one\;
      }{
        go back to the beginning of current section\;
      }
    }
    \caption{How to write algorithms}
  \end{algorithm}
  \subsection{Result JSON Schema}
  JSON schema describing the returned data structure
  \subsection{Visualization Description}
  description of the associated visualization in english
  \subsection{Visualization prototype}
  This section will be updated to a prototype viz

  \section{Question 2  Name}
intro text for the question

\subsection{Statements}

\subsubsection{Ideal Statements}
paragraph or list describing the ideal input statements

\subsubsection{statement parameters to utilize}

\begin{itemize}
  \item first param
  \item second param
  \item third param
  \end{itemize}

\subsubsection{TLA Statement problems}
paragraph talking about known data issues within current TLA
implementation

\subsection{Algorithm}

\subsubsection{Summary}
\begin{enumerate}
  \item step 1
  \item step 2
  \item step 3
  \end{enumerate}

\subsection{Z Specification}

\subsubsection{Introduce Basic Types}
\begin{paragraph}{Template}
  [Name of variable(s) of type set]
\end{paragraph}
\begin{paragraph}{Example}
  [X]
\end{paragraph}
\subsubsection{Example Schema}
Basic unit of specification, defines state variables, system state, operations, etc.
\begin{paragraph}{Template}
\begin{schema}{Schema Name}
  Variable Declarations
  \where
  Predicate/Invariants
\end{schema}
\end{paragraph}
\begin{paragraph}{Example}
  \begin{schema}{Counter}
  ctx : \nat
  \where
  0 \leq ctr \leq max
\end{schema}
\begin{paragraph}{Variables}
  \begin{schema}{Counter}
    ctx : \nat
    \where
\end{schema}
\begin{itemize}
\item the variable ctx is a natural number
\end{itemize}
\end{paragraph}
\begin{paragraph}{Predicates}
  \begin{schema}{Counter}
  \where
  0 \leq ctr \leq max
\end{schema}
  \begin{itemize}
  \item ctr is greater than or equal to 0
  \item ctr is less than or equal to max
  \end{itemize}
\end{paragraph}
\end{paragraph}

\subsubsection{Initialisation}
The starting conditions
\begin{paragraph}{Template}
\begin{schema}{Init[VarName]}
  NameOfExistingSchema
  \where
  InitStateOfVarsWithinRefSchema
\end{schema}
\end{paragraph}
\begin{paragraph}{Example}
  \begin{schema}{InitCounter}
    Counter
    \where
    ctr = 0
  \end{schema}
  \begin{itemize}
    \item the value of the counter starts at 0
  \end{itemize}
\end{paragraph}

\subsubsection{Operations}
an operation is specified in Z with a predicate relating the state before and after the invocation of that operation
\begin{paragraph}{Template}
  \begin{schema}{OperationName}
    \Delta SchemaName \\
    inputParam? : SomeType \\
    outputParam! : SomeType
    \where
    InvariantPredicate \\
    NewValForVar' = OperationOnInput/OutputParams
  \end{schema}
\end{paragraph}
\begin{paragraph}{Example}
  \begin{schema}{Increment}
    \Delta Counter
    \where
    ctr < max \\
    ctr' = crt + 1
  \end{schema}
  \begin{itemize}
    \item There is an implicit conjunction (logical-and) between successive lines of the predicate
    \end{itemize}
    \begin{schema}{Decrement}
      \Delta Counter \\
      d? : \nat
      \where
      ctr \geq d? \\
      ctr' = ctr - d?
    \end{schema}
    \begin{itemize}
    \item input params suffixed with ?
    \end{itemize}
    \begin{schema}{Display}
      \Xi Counter \\
      c! : \nat
      \where
      c! = ctr
    \end{schema}
    \begin{itemize}
    \item output params suffixed with !
    \item the greek symbol means that the operation cannot change the state of Counter
    \end{itemize}
  \end{paragraph}
  \subsection{Pseudocode}
  \begin{algorithm}[H]
    \SetAlgoLined
    \KwIn{this text}
    \KwResult{how to write algorithm with \LaTeX2e }
    initialization\;
    \While{not at end of this document}{
      read current\;
      \eIf{understand}{
        go to next section\;
        current section becomes this one\;
      }{
        go back to the beginning of current section\;
      }
    }
    \caption{How to write algorithms}
  \end{algorithm}
  \subsection{Result JSON Schema}
  JSON schema describing the returned data structure
  \subsection{Visualization Description}
  description of the associated visualization in english
  \subsection{Visualization prototype}
  This section will be updated to a prototype viz
\end{document}
