\documentclass[../main.tex]{subfiles}
\begin{document}

\subsection{At JSONPath}
Performs a lookup at $path$ within $source$ similar to atKey
$$atJsonPath(source, path)$$
such that the fundamental functionality of JSONPath is covered in this definition.
\begin{itemize}
\item A more complete definition will come at a future date if/as necessary
\end{itemize}

\subsubsection{Arguments}
\begin{itemize}
\item $source$ is an object Scalar, KV, Statement or an Algorithm State
\item $path$ is a \href{https://goessner.net/articles/JsonPath/index.html#e2}{JSONPath string} which adheres to the
  \href{https://github.com/adlnet/xapi-profiles/blob/master/xapi-profiles-structure.md#81-statement-template-rules}{additional requirements, clarifications, and additions} placed on JSONPath by the \href{https://github.com/adlnet/xapi-profiles/blob/master/xapi-profiles-structure.md#part-two}{xAPI Profile Specification}
\end{itemize}

\subsubsection{Relevant Operations}
The primitive $atJsonPath$ uses the operations
\begin{itemize}
\item atKey
\item atIndex
\item append
\item count
\end{itemize}

\subsubsection{Summary}
$atJsonPath$ will return a $v$ found within $source$ after converting
$$path \to <path_{i+1}..path_{j}>$$
such that if
$$path = \$.a.b$$
then
$$path \to <a, b>$$
so that
$$atJsonPath(<a \mapsto b \mapsto 123>, \$.a.b) = 123$$

\subsubsection{Usage of Operations}
In order to convert
$$path \to <path_{i+2}..path_{j}>$$
an empty Collection $keyState$ is introduced
$$keyState = <>$$
so that the relevant $k$'(s) can be stored in $keyState$ during iteration over $path$
$$\forall n : i..j \ \bullet i = 0 \ \land j = count(path) - 1$$
and the number of stored keys can be tracked using $curKeyStateIndex$
$$curKeyStateIndex = count(keyState) - 1$$
such that the current $path_{n}$ can be retrieved
$$curKey = atIndex(path, n)$$
and $keepKey?$ can indicate the relevance of $path_{n}$
$$keepKey? = true \iff curKey \not= \$ \ \land curKey \not= . $$
such that during each iteration $n$, $keyState$ will be updated if necessary
$$keyState = append(keyState, \ curKey, \ curKeyStateIndex) \iff keepKey? = true$$
so at the end of the loop
$$keyState = <path_{i+2}..path_{n}..path_{j}>$$
which provides the Collection of Key(s) necessary for calling $atKey$
$$valueInSource = atKey(source, keyState)$$
such that
$$atJsonPath(source,path) \equiv atKey(source, keyState)$$

\subsubsection{Example output}
Given an example $source$
$$source = <a \mapsto <b \mapsto 123, c \mapsto 456>, d \mapsto foo>$$
then
$$atJsonPath(source, \$.a) = <b \mapsto 123, c \mapsto 456>$$
and
$$atJsonPath(source, \$.a.b) = 123$$
and
$$atJsonPath(source, \$.a.c) = 456$$
and
$$atJsonPath(source, \$.d) = foo$$

\end{document}
