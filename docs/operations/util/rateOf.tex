\documentclass[../../main.tex]{subfiles}
\begin{document}

\subsection{Rate Of}
The Operation $rateOf$ calculates the number of times something occured
within an interval of time given a unit of time.
$$rateOf(nOccurances, start, end, unit)$$
Where the output translates to: the rate of occurance per unit within interval
\begin{itemize}
\item $nOccurances$ is the number of times something happened and should be an Integer (called $nO?$ bellow)
\item $start$ is an ISO 8601 timestamp which serves as the first timestamp within the interval
\item $end$ is an ISO 8601 timestamp which servers as the last timestamp within the interval
\item $unit$ is a String Enum representing the unit of time
\end{itemize}
This can be seen in the definition of $rateOf$ bellow.
\begin{schema}{RateOf[\nat, TIMESTAMP, TIMESTAMP, TIMEUNIT]}
  nO? : \nat \\
  rate! : \num \\
  start?, end? : TIMESTAMP \\
  unit? : TIMEUNIT \\
  rateOf~\_ : \nat \cross TIMESTAMP \cross TIMESTAMP \cross TIMEUNIT \fun \num
  \where
  rate! = rateOf(nO?, start?, end?, unit?) @ \\
  \t1 let \ \ ~~ interval == isoToUnix(end) - isoToUnix(start) \\
  \t2 unitS == toSeconds(unit?) \\
  \t1 \ = nO? \div (interval \div units)
\end{schema}
The only other functionality required by $rateOf$ is supplied via basic arithmetic
\begin{argue}
  start = 2015-11-18T12:17:00Z ~ \land ~ end = 2015-11-18T14:17:00Z \\
  unit = second \\
  startN = isoToUnix(start) = 1447849020 \\
  endN = isoToUnix(end) = 1447856220 \\
  interval = endN - StartN = 7200\\
  unitN = toSeconds(unit) = 60 \\
  nO? = 10 \\
  rate! = rateOf(nO?, start, end, unit) = 0.001389 \implies 10 \div (7200 \div 60) \\
  rate!! = rateOf(nO?, start, end, hour) = 5 \implies 10 \div (7200 \div 3600)
\end{argue}
\end{document}
