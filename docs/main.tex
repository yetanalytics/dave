\documentclass{article}

%%% Package imports
\usepackage{graphicx}
\graphicspath{{../../resources/plots/} {../resources/plots/} {../../resources/figures/} {../resources/figures/}}
\usepackage{placeins}
\usepackage{subfiles}
\usepackage[utf8]{inputenc}
\usepackage[ruled]{algorithm2e}
\usepackage{hyperref}
\hypersetup{
    colorlinks=true,
    linkcolor=blue,
    filecolor=magenta,
    urlcolor=cyan,
}
\usepackage{amsmath}
\usepackage{zed-csp}
\usepackage{breqn}
\usepackage{xcolor}
\usepackage{listings}
\usepackage{pgfplots}
\usepackage{pgfplotstable}
\pgfplotsset{compat=newest}
\usepgfplotslibrary{dateplot}
\usepgfplotslibrary{polar}
\usetikzlibrary{pgfplots.dateplot}
\usetikzlibrary{pgfplots.patchplots}
\usetikzlibrary{patterns}

\usepackage{floatrow}

\usepackage{calc}
%%% \makeatletter\amparswitchfalse\makeatother\
\DeclareMarginSet{hangleft}{\setfloatmargins
{\hskip-\marginparwidth\hskip-\marginparsep}{\hfil}}
\floatsetup[widefigure]{margins=hangleft}
%%% ^ Figure formatting within Appendex A

\lstset{literate = {-}{-}1} % get dashs to show up

\pgfplotsset{compat=1.15}

\SetKw{KwBy}{by}

\usepackage{titlesec}
\newcommand{\sectionbreak}{\clearpage}

\title{Data Analytics and Visualization Environment for xAPI and the Total Learning Architecture: DAVE Learning Analytics Algorithms}
\author{Yet Analytics}

\begin{document}

\begin{titlepage}
  \maketitle
\end{titlepage}

\section*{Introduction}

This report introduces the updated definition of learning analytics algorithms in terms of
\textbf{Operations}, \textbf{Primitives} and \textbf{Algorithms} and presents an updated definition for
each of the previously defined algorithms. The previous definitions will be included for reference.
In a more general sense, this report establishes a set of style guidelines for the reporting of algorithms and associated visualization templates.

This document will be updated to include additional Operations, Primitives and Algorithms
as they are defined by the Author of this report or members of the Open Source Community.
Updates may also address refinement of existing definitions and this document should be
understood to be an example of algorithm presentation and not the
final state of any defined algorithm.

$\\\\$
The structure of this documents is as follows:
\begin{enumerate}
\item An Introduction to Z notation and its usage in this document
\item A formal specification for xAPI written in Z
\item An Introduction to Terminology of Operations, Primitives and Algorithms
\item What is an Operation
\item What is a Primitive
\item What is an Algorithm
\item Foundational Operations
\item Example Primitives
\item An algorithm definition including
  \begin{enumerate}
  \item Init
  \item Relevant?
  \item Accept?
  \item Step
  \item Result
  \end{enumerate}
\item Previous Algorithm definitions where each consists of
  \begin{enumerate}
  \item an introduction for the algorithm
  \item the structure of the ideal input data
  \item how to retrieve input data from an LRS
  \item the statement parameters which the algorithm will utilize
  \item notices regarding data collected during the 2018 pilot test of
    the TLA
  \item a summary of the algorithm
  \item the formal specification of the algorithm
  \item pseudocode representation of the algorithm
  \item JSONSchema for the output of the algorithm
  \item a description of the associated visualization
  \item a prototype of the visualization
  \item a collection of suggestions describing how the algorithm could be
    adapted to improve the quality of the visualization prototype
  \end{enumerate}
\end{enumerate}

\subfile{z/introduction.tex}
\subfile{z/xapi.tex}
\subfile{algorithms/introduction.tex}

\section{Foundational Operations}
The Operations in this section use the Operations pulled from the Z Reference Manual (section 1,4) within their own definitions.
They are defined as Operations opposed to Primitives because they represent core functionality needed in the context
of processing xAPI data given the definition of an Algorithm above. As such, these Operations are added to the global
dictionary of symbols usable within the definition of Operations and Primitives throughout the rest of this document.
\subsection{Collections}
Operations which expect a Collection $X = \langle x_{i}..x_{n}..x_{j} \rangle$
\subfile{operations/collections/array?.tex}
\subfile{operations/collections/append.tex}
\subfile{operations/collections/remove.tex}
\subfile{operations/collections/atIndex.tex}

\subsection{Key Value Pairs}
Operations which expect a Map $M = \ldata k_{i}v_{k_{i}}..k_{n}v_{k_{n}}..k_{j}v_{k_{j}} \rdata$

\subfile{operations/kv/map?.tex}
\subfile{operations/kv/associate.tex}
% WIP - update remaining KV to Z - last of the Operations
\subfile{operations/kv/dissociate.tex}
\subfile{operations/kv/atKey.tex}

\subsection{Utility}
Operations which are usefull in many Statement processing contexts.
\subfile{operations/util/map.tex}
\subfile{operations/util/isoToUnix.tex}
\subfile{operations/util/timeUnitToNumberOfSeconds.tex}

\section{Common Primitives}
% WIP - update to Z + new Operation definitions
\subfile{primitives/accumulate.tex}
\subfile{primitives/atJsonPath.tex}
\subfile{primitives/rateOf.tex}

\section*{Updated Algorithm Definitions}
% WIP - update to Z and to account for above
The following are examples of the new way in which Algorithms were defined. These sections are either in draft form or are a work in progress.
\subfile{algorithm_definitions/rateOfCompletions.tex}
\subfile{algorithm_definitions/timelineLearnerSuccess.tex}
\subfile{algorithm_definitions/mostDifficultAssessmentQuestions.tex}
\subfile{algorithm_definitions/followedRecommendations.tex}

\section*{Previous Algorithm Definitions}
The following are examples of the previous way in which Algorithms were defined.
\subfile{algorithms/rate_of_completions.tex}
\subfile{algorithms/timeline_learner_success.tex}
\subfile{algorithms/most_difficult_assessment_questions.tex}
\subfile{algorithms/followed_recommendations.tex}

\subfile{appendices/a.tex}


\end{document}
